\documentclass[12pt, twoside]{article}
\usepackage[letterpaper, portrait, left = 1.375in, right = 1.375in, top = 1in, bottom=1.5in]{geometry} 
%\usepackage{setspace} \doublespacing
%\usepackage[letterpaper, portrait, margin=1.3in]{geometry}
\usepackage[table,xcdraw]{xcolor}
\usepackage{amssymb}
\usepackage{amsfonts}
\usepackage{longtable}
\usepackage{amsmath,amsthm}
\usepackage{enumitem}
\usepackage[utf8]{inputenc}
\usepackage{mathtools}
\usepackage{graphicx}
\usepackage{parskip}
\usepackage{hyperref}
\usepackage{multicol}
\usepackage{listings}
\usepackage[skip=0.25pt]{caption}
\usepackage[mathscr]{euscript}



%\usepackage[cmintegrals,cmbraces]{newtxmath}
%\usepackage{ebgaramond-maths}
\usepackage[T1]{fontenc}
%\usepackage{fourier}
%----------FONT OPTIONS----------
% sans-serif
%\usepackage[sfdefault]{FiraSans}
% \usepackage[sfdefault]{roboto}
% \usepackage[sfdefault]{noto-sans}
% \usepackage[default]{sourcesanspro}

% serif
%\usepackage{CormorantGaramond}
%\usepackage{charter}

\usepackage{cleveref}
\definecolor{dg}{RGB}{10, 100, 10}
\setlength{\parindent}{0in}
\renewcommand{\qed}{$\blacksquare$}
\newtheoremstyle{style}{1}{1}{\normalfont}{}{\bfseries}{\\}{0cm}{}
\theoremstyle{style}
\newtheorem{lemma}{Lemma}
\newtheorem{thm}[lemma]{Theorem}
\newtheorem{prop}[lemma]{Proposition}
\newtheorem{cor}[lemma]{Corollary}
\newtheorem{cl}[lemma]{Claim}
\newtheorem{rmk}[lemma]{Remark}
\newtheorem{ex}[lemma]{Example}
\newtheorem{defn}[lemma]{Definition}
\newtheorem{qs}{Question}
\newtheoremstyle{styleS}{}{}{\color{dg}}{}{\color{dg}\bfseries}{. }{0cm}{}
\theoremstyle{styleS}
\newtheorem*{sol}{Solution}
\newtheoremstyle{style1}{}{}{\normalfont}{}{\bfseries}{. }{0cm}{}
\theoremstyle{style1}
\newtheorem{prob}{Problem}[section]
\newtheorem*{prb}{Problem}
\newtheorem*{pf}{Proof}
\newcommand{\norm}[2]{
\left\lVert #1 \right\rVert_{#2}
}
\usepackage{mathrsfs}
\usepackage[table,xcdraw]{xcolor}
\usepackage{booktabs}
\usepackage{tikz}
\usetikzlibrary{matrix}

\newcommand{\fA}{{\mathfrak{A}}}   \newcommand{\fB}{{\mathfrak{B}}}
\newcommand{\fC}{{\mathfrak{C}}}   \newcommand{\fD}{{\mathfrak{D}}}
\newcommand{\fE}{{\mathfrak{E}}}   \newcommand{\fF}{{\mathfrak{F}}}
\newcommand{\fG}{{\mathfrak{G}}}   \newcommand{\fH}{{\mathfrak{H}}}
\newcommand{\fI}{{\mathfrak{I}}}   \newcommand{\fJ}{{\mathfrak{J}}}
\newcommand{\fK}{{\mathfrak{K}}}   \newcommand{\fL}{{\mathfrak{L}}}
\newcommand{\fM}{{\mathfrak{M}}}   \newcommand{\fN}{{\mathfrak{N}}}
\newcommand{\fO}{{\mathfrak{O}}}   \newcommand{\fP}{{\mathfrak{P}}}
\newcommand{\fQ}{{\mathfrak{Q}}}   \newcommand{\fR}{{\mathfrak{R}}}
\newcommand{\fS}{{\mathfrak{S}}}   \newcommand{\fT}{{\mathfrak{T}}}
\newcommand{\fU}{{\mathfrak{U}}}   \newcommand{\fV}{{\mathfrak{V}}}
\newcommand{\fW}{{\mathfrak{W}}}   \newcommand{\fX}{{\mathfrak{X}}}
\newcommand{\fY}{{\mathfrak{Y}}}   \newcommand{\fZ}{{\mathfrak{Z}}}

\newcommand{\cA}{{\mathcal{A}}}   \newcommand{\cB}{{\mathcal{B}}}
\newcommand{\cC}{{\mathcal{C}}}   \newcommand{\cD}{{\mathcal{D}}}
\newcommand{\cE}{{\mathcal{E}}}   \newcommand{\cF}{{\mathcal{F}}}
\newcommand{\cG}{{\mathcal{G}}}   \newcommand{\cH}{{\mathcal{H}}}
\newcommand{\cI}{{\mathcal{I}}}   \newcommand{\cJ}{{\mathcal{J}}}
\newcommand{\cK}{{\mathcal{K}}}   \newcommand{\cL}{{\mathcal{L}}}
\newcommand{\cM}{{\mathcal{M}}}   \newcommand{\cN}{{\mathcal{N}}}
\newcommand{\cO}{{\mathcal{O}}}   \newcommand{\cP}{{\mathcal{P}}}
\newcommand{\cQ}{{\mathcal{Q}}}   \newcommand{\cR}{{\mathcal{R}}}
\newcommand{\cS}{{\mathcal{S}}}   \newcommand{\cT}{{\mathcal{T}}}
\newcommand{\cU}{{\mathcal{U}}}   \newcommand{\cV}{{\mathcal{V}}}
\newcommand{\cW}{{\mathcal{W}}}   \newcommand{\cX}{{\mathcal{X}}}
\newcommand{\cY}{{\mathcal{Y}}}   \newcommand{\cZ}{{\mathcal{Z}}}

\newcommand{\sA}{{\mathscr{A}}}   \newcommand{\sB}{{\mathscr{B}}}
\newcommand{\sC}{{\mathscr{C}}}   \newcommand{\sD}{{\mathscr{D}}}
\newcommand{\sE}{{\mathscr{E}}}   \newcommand{\sF}{{\mathscr{F}}}
\newcommand{\sG}{{\mathscr{G}}}   \newcommand{\sH}{{\mathscr{H}}}
\newcommand{\sI}{{\mathscr{I}}}   \newcommand{\sJ}{{\mathscr{J}}}
\newcommand{\sK}{{\mathscr{K}}}   \newcommand{\sL}{{\mathscr{L}}}
\newcommand{\sM}{{\mathscr{M}}}   \newcommand{\sN}{{\mathscr{N}}}
\newcommand{\sO}{{\mathscr{O}}}   \newcommand{\sP}{{\mathscr{P}}}
\newcommand{\sQ}{{\mathscr{Q}}}   \newcommand{\sR}{{\mathscr{R}}}
\newcommand{\sS}{{\mathscr{S}}}   \newcommand{\sT}{{\mathscr{T}}}
\newcommand{\sU}{{\mathscr{U}}}   \newcommand{\sV}{{\mathscr{V}}}
\newcommand{\sW}{{\mathscr{W}}}   \newcommand{\sX}{{\mathscr{X}}}
\newcommand{\sY}{{\mathscr{Y}}}   \newcommand{\sZ}{{\mathscr{Z}}}

\newcommand{\ta}{{\tilde{a}}}   \newcommand{\tb}{{\tilde{b}}}
\newcommand{\tc}{{\tilde{c}}}   \newcommand{\td}{{\tilde{d}}}
\newcommand{\te}{{\tilde{e}}}   \newcommand{\tf}{{\tilde{f}}}
\newcommand{\tg}{{\tilde{g}}}   
\newcommand{\ti}{{\tilde{i}}}   \newcommand{\tj}{{\tilde{j}}}
\newcommand{\tk}{{\tilde{k}}}   \newcommand{\tl}{{\tilde{l}}}
\newcommand{\tm}{{\tilde{m}}}   \newcommand{\tn}{{\tilde{n}}}
		         	\newcommand{\tp}{{\tilde{p}}}
\newcommand{\tq}{{\tilde{q}}}   \newcommand{\tr}{{\tilde{r}}}
\newcommand{\ts}{{\tilde{s}}}   
\newcommand{\tu}{{\tilde{u}}}   \newcommand{\tv}{{\tilde{v}}}
\newcommand{\tw}{{\tilde{w}}}   \newcommand{\tx}{{\tilde{x}}}
\newcommand{\ty}{{\tilde{y}}}   \newcommand{\tz}{{\tilde{z}}}


\newcommand{\into}{\hookrightarrow}
\newcommand{\onto}{\twoheadrightarrow}
\newcommand\N{\ensuremath{\mathbb{N}}}
%\newcommand\G{\ensuremath{\mathbb{G}}}
%\newcommand\L{\ensuremath{\mathbb{L}}}
\newcommand\M{\ensuremath{\mathbb{M}}}
\newcommand\R{\ensuremath{\mathbb{R}}}
\newcommand\Z{\ensuremath{\mathbb{Z}}}
\renewcommand\O{\ensuremath{\emptyset}}
\newcommand\Q{\ensuremath{\mathbb{Q}}}
\newcommand\C{\ensuremath{\mathbb{C}}}
\newcommand{\K}{\ensuremath{\mathbb{K}}}
\newcommand\F{\ensuremath{\mathbb{F}}}
\newcommand{\aff}{\ensuremath{\mathbb{A}}}
\newcommand{\proj}{\ensuremath{\mathbb{P}}}
\newcommand{\dd}{\mathrm{d}}
\newcommand{\m}{\ensuremath{\mathfrak{m}}}
\newcommand{\p}{\ensuremath{\mathfrak{p}}}
\newcommand{\n}{\ensuremath{\mathfrak{n}}}
\renewcommand{\phi}{\varphi}
\renewcommand{\qedsymbol}{\ensuremath{\blacksquare}}
%\newcommand{\st}{\;|\;}
\newcommand{\st}{%
  \nonscript\;
  \ifnum\currentgrouptype=16
    \;\middle|\;
  \else
    \;|\;
  \fi
  \nonscript\;}
\newcommand{\ltr}{\par \noindent \framebox[1\width]{ $\implies$ } \hspace{.2cm}}
\newcommand{\rtl}{\par \noindent \framebox[1\width]{ $\impliedby$ } \hspace{.2cm} }
\newcommand{\abs}[1]{\left| #1 \right|}
\newcommand{\inner}[2]{\left\langle #1, #2 \right\rangle}
\newcommand*\circled[1]{\tikz[baseline=(char.base)]{
            \node[shape=circle,draw,inner sep=2pt] (char) {#1};}}

\DeclareMathOperator{\sym}{Sym}
\DeclareMathOperator{\Tor}{Tor}
\DeclareMathOperator{\Ext}{Ext}
\DeclareMathOperator{\adj}{adj}
\DeclareMathOperator{\Tr}{Tr}
\DeclareMathOperator{\GL}{GL}
%\DeclareMathOperator{\tr}{tr}
\DeclareMathOperator{\orbit}{Or}
\DeclareMathOperator{\stab}{Stab}
\DeclareMathOperator{\fix}{Fix}
\DeclareMathOperator{\re}{Re}
\DeclareMathOperator{\im}{Im}
\DeclareMathOperator{\ord}{Ord}
\DeclareMathOperator{\mspec}{mSpec}
\DeclareMathOperator{\spec}{Spec}
\DeclareMathOperator{\frob}{Frob}
\DeclareMathOperator{\id}{Id}
\DeclareMathOperator{\colim}{colim}
\DeclareMathOperator{\loc}{loc}
\DeclareMathOperator{\res}{Res}
\DeclareMathOperator{\rad}{rad}
\DeclareMathOperator{\Res}{Res}
\DeclareMathOperator{\diam}{diam}
\DeclareMathOperator{\arcsec}{arcsec}
\DeclareMathOperator{\arccot}{arccot}
\DeclareMathOperator{\len}{len}
\DeclareMathOperator{\area}{area}
\DeclareMathOperator{\vol}{vol}
\DeclareMathOperator{\ev}{ev}
\DeclareMathOperator{\sgn}{sgn}
\DeclareMathOperator{\supp}{supp}
\DeclareMathOperator{\diff}{d}
\DeclareMathOperator{\Dom}{Dom}
\DeclareMathOperator{\rk}{rank}
\renewcommand{\d}{\diff}
\let\oldend\endlinechar
\renewcommand{\endlinechar}{\oldend}
\newcommand{\open}{\underset{\text{open}}{\subset}}
\newcommand{\divides}{\mathbin{|}}
\newcommand{\set}[1]{\ensuremath{\left\{#1\right\}}}
\newcommand{\sett}{\coloneqq}
\newcommand*\isomap{%
  \xrightarrow{\raisebox{-0.9ex}[0ex][0ex]{$\sim$}}%
}
\renewcommand{\epsilon}{\varepsilon}
\renewcommand\intercal{{\cramped{{}^\mathsf{T}}}}
%\fontfamily{qcr}\selectfont 
%\usepackage[backend=bibtex]{biblatex}
\setcounter{tocdepth}{3}
\usepackage{etoolbox}

\title{Tensor normal models on Directed acyclic graphs}

%\author{Nilava Metya}
\date{}

\begin{document}




\begin{abstract}
Abstract here.
\end{abstract}
\maketitle

\section{Introduction}


Maximum Likelihood Estimation (MLE) is a fundamental problem in statistics, and recently it has been studied from the lens of algebra. The MLE problem is to find a point in a model that `best' fits some data. The idea of `fitting' a data is given by a \emph{likelihood function}, and the problem of finding a point that `best' fits the data translates to \emph{maximizing} the likelihood function. A point that maximizes the likelihood function is called the \emph{maximum likelihood estimate}.

Throughout we will use the calligraphic font $\cG$ to denote a directed acyclic graph (DAG) and the normal font $G$ to denote a group. $\cM$, with some subscript containing an object, will be used to denote a model. For example, $\cM_G$ is the model associated to a group $G$. $\PD_m$ will refer to the cone of symmetric $m\times m$ positive definite matrices.


\section{Main Results}


The main paper \cite{mainpaper} deals mainly about Gaussian group models and section $5$ talks about certain DAG models and their relationship with Gaussian group models. Anna Seigal suggested to deal with the model $\cM(\cG_1,\cG_2)\sett \set{\Psi_1\otimes \Psi_2: \Psi_i\in \cM_{\cG_i}}$ when two DAGs $\cG_1,\cG_2$ are given, and raised the primary question that when are such models Gaussian group models, that is, when is $\cM(\cG_1,\cG_2) = \cM_G$ for some group $G$ of matrices. For the one-parameter case, such models have been studied. Given a DAG $\cG$, one considers the set of matrices $$G\left(\cG\right) =  \set{X\in GL_{\abs{V(\cG)}}: X_{ij} = 0 \text{ for } i\neq j \text{ with } j\not \to i \text{ in }\cG}.$$
This is relevant in the `good' case when $\cM_\cG = \cM_{G\left(\cG\right)}$ as indicated in \cref{tdag_thm}. 

The problem on $\cM(\cG_1,\cG_2)$ naturally starts by considering $$G\left(\cG_1,\cG_2\right) \sett \set{\Psi_1\otimes\Psi_2: \Psi_i\in G\left(\cG_i\right)} = G\left(\cG_1\right)\otimes G\left(\cG_2\right).$$ It can be directly proven that 
\begin{prop}
$G(\cG_1,\cG_2)$ is a group iff both $\cG_i$ are TDAGs. If both $\cG_i$'s are TDAGs, the model $M = \set{\Psi_1\otimes\Psi_2: \Psi_i\in \cM_{\cG_i}}$  is exactly $\cM_{G\left(\cG_1,\cG_2\right)}$.
\end{prop}

My main contribution is to define a construction $\cG_1\otimes \cG_2$ such that $G\left(\cG_1\right)\otimes G\left(\cG_2\right) \cong G\left(\cG_1\otimes \cG_2\right)$. In fact such a construction is commutative (upto relabelling) and associative. Thus it extends to $\cG_1\otimes\cdots\otimes \cG_n$ so that $\displaystyle \bigotimes_{i=1}^nG\left(\cG_i\right) \cong  G\left(\bigotimes_{i=1}^n\cG_i\right)$. That is, this construction extends to the tensor normal models on $n$ DAGs.


\section{Background}

\subsection{Maximum likelihood estimation}

We will focus on multivariate Gaussian distributions of mean zero and covariance matrix $\Sigma$, whose density is given by $$f_{\Sigma}(\pmb y) = \frac{1}{\sqrt{\det(2\pi \Sigma)}} \exp\left(-\frac{1}{2}\pmb y^\intercal\Sigma^{-1}\pmb y\right)$$
where $\pmb y\in \R^m$ and $\Sigma \in \PD_m$.
The corresponding $\Psi = \Sigma^{-1}$ will be its \emph{concentration matrix}. A \emph{Gaussian model} is a set of $m-$dimensional Gaussian distributions with mean zero. Such a model is given by a set of $m\times m$ symmetric positive definite covariance matrices. Equivalently they are also determined by a set of concentration matrices in $\PD_m$. We will refer to this as the Gaussian model, instead of the set of densities themselves. So a Gaussian model $\cM$ is just a subset of $\PD_m$ whose elements are to be thought of as concentration matrices.

A maximum likelihood estimate is a point $\Psi$ in the model which maximizes the likelihood of observing some sample data point $\vec {\pmb Y} = (\pmb y_1,\cdots, \pmb y_n)$. Mathematically, we want a $\Psi\in \cM$ that maximizes the likelihood function $\displaystyle L_{\vec {\pmb Y}}(\Psi) = \prod_{i=1}^n f_{\Psi^{-1}}(\pmb y_i)$. Often, it is easier to maximize $l_{\vec {\pmb Y}} \sett \log L_{\vec {\pmb Y}}$ instead of $L_{\vec {\pmb Y}}$ itself, and they have the same maximizers. Note that
\begin{align*}
l_{\vec {\pmb Y}}(\Psi) &= \log L_{\vec {\pmb Y}}(\Psi)\\
&= \sum_{i=1}^n \left(-\frac{1}{2}\log \det(2\pi \Psi^{-1}) - \frac{1}{2} \pmb y_i^\intercal \Psi \pmb y_i\right)\\
\implies \frac{2}{n}l_{\vec {\pmb Y}} &= -\log \det(2\pi \Psi^{-1}) - \frac{1}{n}\sum_{i=1}^n \pmb y_i^\intercal \Psi \pmb y_i\\
&= -m\log (2\pi) + \log \det \Psi - \frac{1}{n}\sum_{i=1}^n \pmb y_i^\intercal \Psi \pmb y_i\\
\implies \frac{2}{n}l_{\vec {\pmb Y}} + m\log(2\pi) &= \log \det \Psi -  \frac{1}{n}\sum_{i=1}^n \Tr\left(\pmb y_i^\intercal \Psi \pmb y_i\right)\\
&= \log \det \Psi -  \Tr\left(\frac{1}{n}\sum_{i=1}^n \Psi\pmb y_i^\intercal \pmb y_i\right)\\
&= \log \det \Psi -  \Tr\left(\Psi \frac{1}{n}\sum_{i=1}^n \pmb y_i\pmb y_i^\intercal\right)\\
&= \log \det \Psi -  \Tr\left(\Psi S_{\vec{\pmb Y}}\right)
\end{align*}
where $\displaystyle S_{\vec{\pmb Y}} = \frac{1}{n}\sum_{i=1}^n \pmb y_i \pmb y_i^\intercal$ is the sample covariance matrix. This is clearly symmetric positive semi-definite. The above calculation shows that $\displaystyle \underset{\Psi\in\cM}{\argmax}\set{l_{\vec{\pmb Y}} (\Psi)} = \underset{\Psi\in\cM}{\argmax}\set{\log \det \Psi -  \Tr\left(\Psi S_{\vec{\pmb Y}}\right)}$. Take $\ell_{\vec{\pmb Y}}(\Psi)\sett \log\det\Psi - \Tr(\Psi S_{\vec{\pmb Y}})$. Observe that $\ell_{\vec{\pmb Y}}(\Psi)$ is invariant under similarity of $\Psi$ and similarity of $S_{\vec{\pmb Y}}$. Further $\Psi,S_{\vec{\pmb Y}}$ are real symmetric matrices, hence have real eigenvalues and are diagonalizable. This means that if $\set{\lambda_i}_{i=1}^m \subseteq \R^{>0}$ and $\set{\mu_i}_{i=1}^m\subseteq \R^{\geq 0}$ are eigenvalues of $\Psi$ and $S_{\vec{\pmb Y}}$ respectively, then $\ell_{\vec{\pmb Y}}(\Psi) = \sum\log \lambda_i - \sum \lambda_i\mu_i$. Note that if $S_{\vec{\pmb Y}}$ is invertible, then 
\begin{align*}
\ell_{\vec{\pmb Y}}(\Psi) & = \log \det \Psi - \Tr\left(\Psi S_{\vec{\pmb Y}}\right)\\
&= \log \det \Psi - \log \det S_{\vec{\pmb Y}}^{-1} + \log \det S_{\vec{\pmb Y}}^{-1} \\
&\qquad \quad- \Tr\left(\Psi S_{\vec{\pmb Y}}\right) + \Tr \pmb 1_m - \Tr\left( S_{\vec{\pmb Y}}^{-1} S_{\vec{\pmb Y}}\right)\\
&= \sum\log (\lambda_i\mu_i) + \log \det S_{\vec{\pmb Y}}^{-1} - \sum \lambda_i\mu_i + m - \Tr\left( S_{\vec{\pmb Y}}^{-1} S_{\vec{\pmb Y}}\right)\\
&= \sum \left[\log (\lambda_i\mu_i) - \lambda_i\mu_i\right] + m + \ell_{\vec{\pmb Y}} \left(S_{\vec{\pmb Y}}^{-1}\right)\\
&\leq \sum_{i=1}^m(-1) + m + \ell_{\vec{\pmb Y}} \left(S_{\vec{\pmb Y}}^{-1}\right) = \ell_{\vec{\pmb Y}} \left(S_{\vec{\pmb Y}}^{-1}\right)
\end{align*}
If $S_{\vec{\pmb Y}}$ is singular, assume WLOG $\mu_1=0$, then $ \ell_{\vec{\pmb Y}} \left(\Psi\right) = \log\lambda_1 + \sum\limits_{i\geq 2} \left(\log \lambda_i - \lambda_i\mu_i\right)$ diverges to $+\infty$ as $\lambda_1\to \infty$.



\subsection{Gaussian group models}

The \emph{Gaussian group model} given by a group $G \subseteq \GL(\R^n)$ is the multivariate Gaussian model comprising all densities of mean zero and concentration matrices given by $$\cM_G\sett \set{X^\intercal X: X\in G}.$$ The importance of such models is that finding an MLE is equivalent to an optimization problem. This is seen by the following calculation
\begin{align}
-\ell_{\vec{\pmb Y}}\left(X^\intercal X\right) &= \frac{1}{n}\sum\Tr \left(\left(X\pmb y_i\right)^\intercal X\pmb y_i\right) - \log\det \left(X^\intercal X\right)\\
\label{norm_likelihood}&= \frac{1}{n} \sz{X\cdot \vec{\pmb Y}}_2^2 - \log(\det X)^2
\end{align}
So maximizing $\ell_{\vec{\pmb Y}}\left(X^\intercal X\right)$ is equivalent to minimizing $\frac{1}{n} \sz{X\vec{\pmb Y}}_2^2 - \log\det \left(X^\intercal X\right)$ which consists of minimizing norms. The exact statement is captured in the following theorem in \cite[Proposition 3.4]{mainpaper}:
\begin{prop}
Let $\vec{\pmb Y}\in V^n$ be a tuple of samples. If the group $G\subseteq \GL(V)$ is closed under non-zero scalar multiples, the supremum of the log-likelihood $\ell_{\vec{\pmb Y}}(\Psi)$ over $\cM_G$ is the double infimum $$-\inf_{\lambda>0}\left( \frac{\lambda}{n} \left( \inf_{H\in G_{\SL}^{\pm}} \sz{H\cdot \vec{\pmb Y}}^2 \right) - \dim (V) \log \lambda\right).$$
The MLEs, if they exist, are the matrices $\displaystyle \frac{n \dim V}{\sz{H\cdot \vec{\pmb Y}}^2} H^\intercal H$, where $H$ minimizes $\sz{H\cdot \vec{\pmb Y}}$ under the action of $G_{\SL}^\pm$ on $V^n$.
\end{prop}
In the above, $G_{\SL}^\pm = \set{X\in G: \det X = \pm 1}$. 

What the above proposition says is that $\ell_{\vec{\pmb Y}}\left(X^\intercal X\right)$ can be maximized in two steps:
\begin{enumerate} 
\item Minimize the norm $\sz{H\cdot \vec{\pmb Y}}^2$ over $H\in G_{\SL}^\pm$.
\item Find a scalar $\mu\in \R$ so that $X\sett \mu H$ mimimizes $-\ell_{\vec{\pmb Y}}\left(X^\intercal X\right)$ in \cref{norm_likelihood}.
\end{enumerate}

\begin{proof}
We want to minimize the function $f: G \to \R$ given by $$f(X) = \frac{1}{n} \sz{X\cdot \vec{\pmb Y}}_2^2 - \log\left(\det X\right)^2.$$
Let $m=\dim V$. Observe that $f|_{G_{\SL}^\pm}$ determines $f$ completely. This is because for any $X\in G$, take $\displaystyle Z \sett \frac{1}{\mu_X} X$, where $\mu_X = \left(\abs{\det X}\right)^{\frac{1}{m}}\in \R_{>0}$. Clearly $\det Z = \frac{\det X}{\abs{\det X}} = \pm1$. Then \begin{align*}
f(X) &= \frac{\mu_X^2}{n}\sz{Z\cdot \vec{\pmb Y}}_2^2 - \log\left( \left(\mu_X^m\det Z\right)^2\right)\\ 
&= \frac{\mu_X^2}{n}\sz{Z\cdot \vec{\pmb Y}}_2^2 - 2m\log\mu_X\\
&= \mu_X^2 f(Z) - 2m \log \mu_X
\end{align*}
Note that for $K>0$, the function $s \mapsto s K - \log s$ minimizes at $s=\frac{1}{K}$ giving a min value of $1+\log K$, the latter being an increasing function of $K$. Thus minimizing $f(X)$ is equivalent to first minimizing the norm in the orbit of $G_{\SL}^\pm$ and then minimizing the overall expression with the previous minima. In other words,
$$\inf _{X\in G}f(X) = \inf_{\mu>0} \left( \mu^2 \cdot \inf_{Z\in G_{SL}^\pm} \sz{Z\cdot\vec{\pmb Y}}_2^2 - m\log \mu^2\right) .$$
Replacing $\lambda = \mu^2$ gives the expression in the expression in the proposition. The MLE, if it exists, is given by the point $\hat X = \hat\mu \hat Z$, where $\hat Z$ minimizes $\sz{Z\cdot \vec{\pmb Y}}_2^2$ and $\displaystyle \hat \mu = \frac{\sqrt {mn}}{\sz{\hat Z\cdot \vec{\pmb Y}}_2}$, which corresponds to the input $\hat\Psi = \hat X^\intercal \hat X = \hat \mu^2 \hat Z^\intercal \hat Z$.
\hfill\end{proof}


\subsection{Transitive DAGs} 

The relevance of transitive DAGs in the statistical context was introduced in \cite[\S5]{mainpaper}. We briefly introduce the relevant details here. Every DAG $\cG$ has a model associated to them given by $$\cM_{\cG} \sett  \set{(\pmb1-\Lambda)^\intercal\Omega^{-1}(\pmb1-\Lambda) : \Lambda, \Omega\in \R^{m_1\times m_i}, \Lambda_{ij}\neq 0\implies j\to i, \Omega \text{ diagonal and PD}}.$$ Note that taking $\Omega = \pmb 1, \Lambda=\pmb 0$ forces $\pmb 1\in \cM_{\cG}$. These matrices $\Psi\in \cM_{\cG}$ are to be thought of as concentration matrices, so they define a model $\set{f_{\Psi^{-1}}}$ (which is just a collection of probability densities) given by  $$f_{\Sigma}(y) = \frac{1}{\sqrt{\det(2\pi \Sigma)}} \exp\left(-\frac{1}{2}y^\intercal\Sigma^{-1}y\right)$$
where $\Sigma=\Psi^{-1}$. 

$\cG$ is said to be a \emph{transitive} DAG (TDAG) if $i\to k$ is an edge in $\cG$ whenever $i\to j, j\to k$ are. They turn out to be the `good' DAGs that help relate these models to Gaussian group models due to the following proposition in \cite{mainpaper}:

\begin{theorem}\label{tdag_thm}
$G\left(\cG\right)$ is a group iff $\cG$ is a TDAG. In such a case, $\cM_\cG = \cM_{G\left(\cG\right)}$.
\end{theorem}

\subsection{ss}


\bibliographystyle{plain}
\bibliography{refs.bib}


\end{document}
